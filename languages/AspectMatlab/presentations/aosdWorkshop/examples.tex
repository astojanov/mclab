\begin{frame}{Why Aspect Matlab?}
  \begin{onlyenv}<1>
    An aspect extension to Matlab seems odd:
    \begin{itemize}
    \item matlab programs small, scientific computing doesn't rely on classes
    \item cross cutting concerns is not an issue
    \end{itemize}
  \end{onlyenv}
  \begin{onlyenv}<2-3>
    % i was basically given the existing language and compiler and told
    % "hey, you've taking some numerical classes, figure out how this is useful."
     
    % let's look at matlab's domain...
    \begin{itemize}
    \item matlab has some domain
      \begin{itemize}
      \item numerical/scientific computing
      \end{itemize}
    \end{itemize}
  \end{onlyenv}
  \begin{onlyenv}<3>
    % ...and what are the core issues to programmers
    \begin{itemize}
    \item issues:
      \begin{itemize}
      \item performance
        \begin{itemize}
        \item matlab's libraries are fast, and the language is easy
        \item but there are still many tricks to vastly improve
          performance
        \end{itemize}
      \item domain specific utilities
        \begin{itemize}
        \item thousands of extensions
        \end{itemize}
      \end{itemize}
    \end{itemize}
  \end{onlyenv}
  \begin{onlyenv}<4>
    aspect matlab has some domain
    \begin{itemize}
    \item leverage these domains (performance, extensions)
      \begin{itemize}
      \item reusable aspects that help increase performance
        by domain specific profiling
      \end{itemize}
    \item add domain specific utilities
      \begin{itemize}
      \item we can use aspects to extend functionality
      \end{itemize}
    \end{itemize}
  \end{onlyenv}
\end{frame}


\subsection*{Profiling}
% intro
%  profiling
% matlab has fast libraries, but the language has a lot of bottlenecks
% having a lot of detailed knowledge about one's program can help speed up performance
% the aspect matlab language allows for very detailed pointcuts
% before, around, after calling functions, setting and getting arrays
% and provides a lot of context information for every shadow (?)
% -- more on that in the actual talk on x

% we can write reusable aspects that profile very specific properties
% of a program
% examples:
\begin{frame}{Profiling}
  \begin{itemize}
\item pointcuts get/set/call % allow very detailed view of whats going on
\item a lot of context info  %all the context information provides even more info at the same time
\item performance is important in this domain
\begin{itemize}
\item more knowledge improves performance 
\end{itemize}
%\item --> it's a domain where you want to know exactly what's going on in your program
\end{itemize}
\end{frame}
% flops 1
\begin{frame}[fragile]{Profiling I - flops}
  \begin{onlyenv}<1>
    % let's say as a scientist/engineer i want to know where most
    % compuations happens
    % since my program is all matrizes, most of the program is spent doing
    % matrix operations
    % the basic unit of computational complexity is the floating point
    % operation -- tend to be the most numerous and expensive part of a
    % program -- numerical algorithms are counted and analyzed in flops
    \begin{itemize}
     \item where does computation happen?
    \item most of the program - matrix computations
    \item basic unit of computational complexity - flop
      \begin{itemize}
      \item most expensive and numerous
      \item we count counted/analyze programs in flops
      \end{itemize}
    \item goal: aspect that tells us the total flops
    \end{itemize}
  \end{onlyenv}
    \begin{columns}
      \begin{column}{5cm}
        \begin{onlyenv}<2>
          \begin{Verbatim}[commandchars=@\[\]]
aspect flops
  ...
  patterns
    pplus : call(plus)
    pmul  : call(mtimes)
    ...
  end

  actions
    ...
  end
end
        \end{Verbatim}
      \end{onlyenv}
        \begin{onlyenv}<3>
          \begin{Verbatim}[commandchars=@\[\]]
actions
  around pplus : 
    this.flops = this.flops + ..
      numel(args{1})
  end
        
  before pmtimes : (args)
    [m,n] = size(args{1})
    [n,k] = size(args{2})
    this.flops = this.flops+2*m*n*k
  end
end
        \end{Verbatim}
      \end{onlyenv}
      \end{column}
      \begin{column}{5cm}
        \begin{itemize}
          \begin{onlyenv}<2-3>
          \item we catch all operations
          \item in a before we add estimated flops
          \item we display result at the end
            % this can be done in core matlab - but involves some
            % hackery. also aspects allows this to be in one file,
            % much shorter and easier
          \end{onlyenv}
        \end{itemize}
      \end{column}
    \end{columns}


  % so we want to write an aspect that tells us how many flops
  % every operation takes

  % start left/right
  % we simply overrade all operations (show aspect code overriding operations)
  % and in the operations we can add up an estimate for computational of
  % the flops (show advice adding stuff)
\end{frame}

% flops 2
\begin{frame}{Profiling II - flops extended}
  \begin{itemize}
  \item supposse now we want to know where the flops occur, with
    enclosing info like in a profiler
  \item example
  \item now we don't only intercept functions, but also function calls
  \item we keep track of the flops in a stack
  \item we need context information
  \item in plain matlab, this is not possible without extensively
  modifying the original code
  \end{itemize}
\end{frame}

% flops 3
%\begin{frame}{Profiling III - interval arithmetic}
%  \begin{itemize}
%  \item lets go deeper into acally changing the original program
%  \item supposse we want upper ad lower bounds on numerical errors
%  (precision)
%  \item we can override all variables to be a structure including the
%  original value and the new annotated information
%  \item diagram (var) --> structure: value, some tag, annotated info
%  ---- should this actually be a class???
%  \item 
%  \end{itemize}
%\end{frame}


\subsection*{Extending Functionality}
\begin{frame}{Extending Functionality}
  extending functionality
  \begin{itemize}
  \item we have an extensible toolkit (McLab)
  \item we can use aspects for rapid prototyping of new functionality
  \item we can use aspects for writing libraries
  \end{itemize}
\end{frame}

\begin{frame}[fragile]{Extending Functionality I}
  \begin{columns}
    \begin{column}{5cm}
          \begin{onlyenv}<1>
        \begin{Verbatim}[commandchars=@\[\]]
function y=plotThis(f,x)
  y = empty
  for t = x
    y = [y; f(t)]
  end
  plot(x,y)
end
        \end{Verbatim}
          \end{onlyenv}
          \begin{onlyenv}<2>
        \begin{Verbatim}[commandchars=@\[\]]
function y=plotThis(f,x)
  y = x;
  for [t,i] = x
    y(i) = f(t);
  end
  plot(x,y)
end
        \end{Verbatim}
          \end{onlyenv}
          \begin{onlyenv}<3>
        \begin{Verbatim}[commandchars=@\[\]]
function y=plotThis(f,x)
  y = x;
  for t = x
    y(iteration) = f(t);
  end
  plot(x,y)
end
        \end{Verbatim}
          \end{onlyenv}

      \end{column}
      \begin{column}{5cm}
      \begin{itemize}
        \begin{onlyenv}<1-2>
        \item consider our original example
        \item we don't want to grow the array
        \end{onlyenv}
        \begin{onlyenv}<2>
        \item now we need to now i
        \item we propose a language extension 
          \begin{itemize}
          \item simply add iteration counter as second variable
          \end{itemize}
        \end{onlyenv}
        \begin{onlyenv}<3>
        \item implement via aspect
        \item capture all loops
        \item also capture calls to 'iteration'
        \item store context info in loop advice
        \item return that in around 'iteration'
        \item 'iteration' will give us the count
        \end{onlyenv}
      \end{itemize}
    \end{column}
  \end{columns}
\end{frame}




\begin{frame}[fragile]{Extending Functionality II}
\begin{itemize}
\item unit benchmark
\item consider new addition of units
\end{itemize}
\begin{lstlisting}[language=Matlab]
bmi = 180*lb/(5*feet + 8*inches)^2
t = AU/c
\end{lstlisting}
\begin{itemize}
 \item we have some functions, but they affect all the data types.
 \item we have to (override) annotate all data
 \item we need to override loops as well
\end{itemize}
\begin{lstlisting}
aspect unit
properties
units = struct(...
  'm',  [1, 0, 0, 0, 0, 0, 0],...
  'Kg', [0, 1, 0, 0, 0, 0, 0],...
  's',  [0, 0, 1, 0, 0, 0, 0],...
...
  'J',  [2, 1,-2, 0, 0, 0, 0],...
  'N',  [1, 1,-2, 0, 0, 0, 0]);
end

patterns
allCalls : call(*());
...
end

acalls : around allCalls : (name)
if (isfield(this.units,name))
...
end
\end{lstlisting}

\end{frame}

\begin{frame}{Domain Specific Outlook}
\begin{itemize}
  \item several of our aspects override variables
    \begin{itemize}
    \item attach extra information to them
    \end{itemize}
  \item idea: incorporate this theme into a new class of aspects
  \item one could have an aspect that annotates existing variables
  \item operations should be specified on the annoations
  \item a sort of abstract interpretation of annotations at runtime
\end{itemize}
\end{frame}

