%\subsection{Profiling}
% intro
** Use cases

*** profiling
**** overview
     - extra pointcuts allow very detailed view of whats going on
     - all the context information provides even more info
       at the same time
     - scientific programming is basically matrix computations
     - performance is of paramount importance
     - also, knowing all sorts of other properties of your programs helps
       --> it's a domain where you want to know exactly what's going on in your program
% flops 1
\begin{frame}{Profiling I - flops}
  \begin{itemize}
  \item in scientific/numerical computing, computational complexity is
  counted in flops
  \item knowing how many flops are made is useful
  \item we can intercept all calls to operations and accumulate
  estimates for flops
  \item (example for estimating flops of mul?) ... example code?
  \item this could be done with just plain Matlab
  \item aspect matlab allows us to collect all this functionatliay
  (concern?) into one single aspect
  \end{itemize}
\end{frame}

% flops 2
\begin{frame}{Profiling II - flops extended}
  \begin{itemize}
  \item supposse now we want to know where the flops occur, with
  enclosing info like in a profiler
  \item example
  \item now we don't only intercept functions, but also function calls
  \item we keep track of the flops in a stack
  \item we need context information
  \item in plain matlab, this is not possible without extensively
  modifying the original code
  \end{itemize}
\end{frame}

% flops 3
\begin{frame}{Profiling III - interval arithmetic}
  \begin{itemize}
  \item lets go deeper into acally changing the original program
  \item supposse we want upper ad lower bounds on numerical errors
  (precision)
  \item we can override all variables to be a structure including the
  original value and the new annotated information
  \item diagram (var) --> structure: value, some tag, annotated info
  ---- should this actually be a class???
  \item 
  \end{itemize}
\end{frame}


%\subsection{Extending Functionality}

*** extending functionality
    - we have an extensible toolkit (McLab)
    - we can use aspects for rapid prototyping of new functionality
    - we can use aspects for writing libraries

\begin{frame}{Extending Functionality I}
***** case 1)
      add some functions
      - really simple, could be done with just dumping some .m files in the
        same dir
\end{frame}
\begin{frame}{Extending Functionality II}
***** case 2)
      iteration space tricks
      consier new sntax addition
      for [x,i] = Z
      ...
      end

      - more complicated - functions expose context information

\end{frame}
\begin{frame}{Extending Functionality III}
***** case 3)
      unit benchmark
      consider new addition of units
      x = 3*m
      ... (example from example)

      - we have some functions, but they affect all the data types. we have
        to annotate all data
      - we need to override loops as well

\end{frame}

