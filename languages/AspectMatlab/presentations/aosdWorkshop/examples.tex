\begin{frame}{Why Aspect Matlab?}
  \begin{onlyenv}<1-2>
    An aspect extension to Matlab seems odd
    \begin{itemize}
    \item Programs are small % even with libraries
    \item Classes not really used
    \end{itemize}
    \uncover<2>{
      \vspace{4ex}
      \begin{itemize}
      \item What is the domain of Aspect Matlab?
      \end{itemize}  
    }
  \end{onlyenv}
  \begin{onlyenv}<3-4>
    % i was basically given the existing language and compiler and told
    % hey, you've taking some numerical classes, figure out how this is useful.     
    % let's look at matlab's domain...
    \begin{itemize}
    \item Matlab's domain:
      \begin{itemize}
      \item Numerical/scientific computing
      \end{itemize}
    % ...and what are the core issues to programmers
    \uncover<4>{
    \item Issues
      \begin{itemize}
      \item Performance/high productivity
        \begin{itemize}
        \item Fast libraries, easy language
        \item Dynamic features may cause slow execution
        \end{itemize}
      \item Domain-specific extra functionality
        \begin{itemize}
        \item Hundreds of toolboxes
        \end{itemize}
      \end{itemize}
    }
    \end{itemize}
  \end{onlyenv}
  \begin{onlyenv}<5>
    Aspect matlab domain
    \begin{itemize}
    \item Leverage these domains (performance, extensions)
    \item We can rewrite reusable aspects that
      \begin{itemize}
      \item Help increase performance by domain-specific profiling
      \item Add domain-specific functionality
      \end{itemize}
    \end{itemize}
  \end{onlyenv}
\end{frame}


\subsection*{Profiling}
% intro
%  profiling
% matlab has fast libraries, but the language has a lot of bottlenecks
% having a lot of detailed knowledge about one's program can help speed up performance
% the aspect matlab language allows for very detailed pointcuts
% before, around, after calling functions, setting and getting arrays
% and provides a lot of context information for every shadow (?)
% -- more on that in the actual talk on x

% we can write reusable aspects that profile very specific properties
% of a program
% examples:
\begin{frame}{Profiling Aspects}



  \begin{itemize}
\item Pointcuts get/set/call % allow very detailed view of whats going on
\item A lot of context info  %all the context information provides even more info at the same time
\item Performance is important in this domain
\begin{itemize}
\item More knowledge improves performance
\end{itemize}
%\item --> it's a domain where you want to know exactly what's going on in your program
\end{itemize}
\end{frame}
% flops 1
\begin{frame}[fragile]{Profiling I - Flops}
  \begin{onlyenv}<1-4>
    % let's say as a scientist/engineer i want to know where most
    % compuations happens
    % since my program is all matrizes, most of the program is spent doing
    % matrix operations
    % the basic unit of computational complexity is the floating point
    % operation -- tend to be the most numerous and expensive part of a
    % program -- numerical algorithms are counted and analyzed in flops
    \vspace{4ex}
    \begin{itemize}
    \item Where does computation happen?
    \pause \item Most of the program - matrix computations
    \pause \item Basic unit of that - \textbf{flop}
      \begin{itemize}
      \item Most expensive and numerous
      \item We count/analyze programs in flops
      \end{itemize}
    \pause \item Goal: an aspect that tells us the total flops
    \end{itemize}
  \end{onlyenv}
    \begin{columns}
      \begin{column}[T]{5cm}
        \begin{onlyenv}<5>
          \begin{Verbatim}[commandchars=@\[\]]
aspect flops
  ...
  patterns
    pplus : call(plus)
    pmul  : call(mtimes)
    ...
  end

  actions
    ...
  end
end
        \end{Verbatim}
      \end{onlyenv}
        \begin{onlyenv}<6>
          \begin{Verbatim}[commandchars=@\[\]]
aspect flops
  ...
  patterns
    @PYaG[pplus : call(plus)]
    @PYaG[pmul  : call(mtimes)]
    ...
  end

  actions
    ...
  end
end
        \end{Verbatim}
      \end{onlyenv}
        \begin{onlyenv}<7>
          \begin{Verbatim}[commandchars=@\[\]]
actions
  before pplus : 
    @PYaG[this.flops = this.flops + ..]
      @PYaG[numel(args{1})]
  end
        
  before pmtimes : (args)
    [m,n] = size(args{1})
    [n,k] = size(args{2})
    @PYaG[this.flops = this.flops+2*m*n*k]
  end
end
        \end{Verbatim}
      \end{onlyenv}
        \begin{onlyenv}<8>
          \begin{Verbatim}[commandchars=@\[\]]
actions
  around pplus : 
    this.flops = this.flops + ..
      numel(args{1})
  end
        
  before pmtimes : (args)
    @ob[]m,n@cb = size(args{1})
    @ob[]n,k@cb = size(args{2})
    this.flops = this.flops+2*m*n*k
  end
end
        \end{Verbatim}
      \end{onlyenv}
      \end{column}
      \begin{column}[T]{5cm}
        \begin{itemize}
          \pause \item We catch all operations
          \pause \pause \item In befores we add up estimated flops
          \pause \item We display result at the end
            % this can be done in core matlab - but involves some
            % hackery. also aspects allows this to be in one file,
            % much shorter and easier
        \end{itemize}
      \end{column}
    \end{columns}


  % so we want to write an aspect that tells us how many flops
  % every operation takes

  % start left/right
  % we simply overrade all operations (show aspect code overriding operations)
  % and in the operations we can add up an estimate for computational of
  % the flops (show advice adding stuff)
\end{frame}

% flops 2
\begin{frame}[fragile]{Profiling II - Flops Extended}
  \begin{onlyenv}<1>
    \begin{itemize}
    \item Suppose now we want to know where these flops occur
    \item For every function, we want to know
      \begin{itemize}
      \item All flops that occur inside it
      \item How often it gets called
      \end{itemize}
    \end{itemize}
  \end{onlyenv}
  \begin{onlyenv}<2>
\begin{Verbatim}
   'call site'                'number of calls' 'total flops'
   'bar_3_mtimes                 [        40] [       4000]
   'bar_5_plus                   [        20] [        200]
   'foo_3_mtimes'                [         5] [        800]
   'foo_4_bar'                   [         5] [       5000]
   'Script_1_foo'                [         1] [       5000]
\end{Verbatim}
  \end{onlyenv}
  \begin{onlyenv}<3->
    \begin{itemize}
    \pause
    \pause \item Now we don't only intercept Operators, but also function calls
    \pause \item We keep track of the flops in a stack
    \pause \item Before any function call we push 0
    \pause \item Around any operation (*,+,\^,..) we
      \begin{itemize}
      \item Add estimated flops to top of stack
      \end{itemize}
    \pause \item After call we pop the value and
      \begin{itemize}
      \item Record the number of flops
      \item Add that number to the new top
      \end{itemize}    
      % in plain matlab, this is not possible without extensively
      % modifying the original code
    \end{itemize}
  \end{onlyenv}
\end{frame}

% flops 3
%\begin{frame}{Profiling III - interval arithmetic}
%  \begin{itemize}
%  \item lets go deeper into acally changing the original program
%  \item supposse we want upper ad lower bounds on numerical errors
%  (precision)
%  \item we can override all variables to be a structure including the
%  original value and the new annotated information
%  \item diagram (var) --> structure: value, some tag, annotated info
%  ---- should this actually be a class???
%  \item 
%  \end{itemize}
%\end{frame}


\subsection*{Extending Functionality}
\begin{frame}{Extending Functionality}
  Extending Functionality
  \begin{itemize}
  \item We have an extensible toolkit (McLab)
  \item We can use aspects for rapid prototyping of domain-specific extensions
  \item We can use aspects to add new functionality % special libraries
  \end{itemize}
\end{frame}

\begin{frame}[fragile,t]{Extending Functionality I}
  \begin{columns}[T]
    \begin{column}{5cm}
      \begin{onlyenv}<2>
        \begin{Verbatim}[commandchars=@\[\]]
function y=plotThis(f,x)
  y = empty
  for t = x
    y = @ob[]y; f(t)@cb[]
  end
  plot(x,y)
end
        \end{Verbatim}
      \end{onlyenv}
      \begin{onlyenv}<3>
        \begin{Verbatim}[commandchars=@\[\]]
function y=plotThis(f,x)
  y = @PYaG[x]
  for t = x
    @PYaG[y(i) = f(t)]
  end
  plot(x,y)
end
        \end{Verbatim}
          \end{onlyenv}
          \begin{onlyenv}<4>
        \begin{Verbatim}[commandchars=@\[\]]
function y=plotThis(f,x)
  y = x
  for t = x
    y(@PYaG[i]) = f(t)
  end
  plot(x,y)
end
        \end{Verbatim}
          \end{onlyenv}
          \begin{onlyenv}<5>
        \begin{Verbatim}[commandchars=@\[\]]
function y=plotThis(f,x)
  y = x
  for @PYaG[@ob[]t,i@cb] = x
    y(@PYaG[i]) = f(t)
  end
  plot(x,y)
end
        \end{Verbatim}
          \end{onlyenv}
          \begin{onlyenv}<6-10>
        \begin{Verbatim}[commandchars=@\[\]]
function y=plotThis(f,x)
  y = x
  for t = x
    y(@PYaG[iteration()]) = f(t)
  end
  plot(x,y)
end
        \end{Verbatim}
          \end{onlyenv}
          \begin{onlyenv}<11>
        \begin{Verbatim}[commandchars=@\[\]]
function y=plotThis(f,x)
  y = x
  for t = x
    y(@PYaG[iteration]) = f(t)
  end
  plot(x,y)
end
        \end{Verbatim}
          \end{onlyenv}
      \end{column}

      \begin{column}{5cm}
        \begin{onlyenv}<1-5>
          \begin{itemize}
          \item<1-5> Let's consider our original example
          \item<2-5> We don't want to grow the array
          \item<4-5> Now we need to know i
          \item<5> We propose this language extension 
          \end{itemize}
        \end{onlyenv}
        \begin{onlyenv}<6-11>
          \begin{itemize}
          \item<6-> Via an aspect we simply add an iteration() function
          \item<7-> Capture all loops
          \item<8-> Store context info in loop advice 
           \item<9-> Return that in an around for a call to 'iteration'
          \item<10> 'iteration' returns the count
          \end{itemize}
        \end{onlyenv}
    \end{column}
  \end{columns}
\end{frame}




\begin{frame}[fragile,t]{Extending Functionality II - Units}
  \begin{itemize}
  \item Let's say we want to have units
  \end{itemize}
  \begin{onlyenv}<2->
    \begin{Verbatim}
      bmi = 180*lb/(5*feet + 8*inches)^2
      v = 374*km / ((16*h+07*min)-(13*h+16*min))
      t = AU/c
    \end{Verbatim}
  \end{onlyenv}
  \begin{itemize}
    \pause
    \pause \item Then we have to
    \begin{itemize}
      \pause \item Annotate (override) all data
      \pause \item Override all operations
      \pause \item Override loops to restore the semantics
      \pause \item Capture all calls to units
    \end{itemize}
  \end{itemize}
\end{frame}


\begin{frame}[fragile,t]{Extending Functionality II - Units}
\begin{columns}[T]
  \begin{column}{10cm}
    \begin{itemize}
      \only{ \item We don't define a pattern for every unit.}<1->
      \only{ \item We simply keep a list of units...}<2->
      \only{ \item ..and constants.}<3->
      \only{ \item We match all calls with 0 arguments.}<4->
      \only{ \item We override them only if they are in the lists.}<5->
      \only{ \item We can still use unit names as variables.}<6->
    \end{itemize}
    \begin{onlyenv}<2>
      \begin{Verbatim}[commandchars=@\[\]]
  units = struct(...
    'm',  @ob1, 0, 0, 0, 0, 0, 0@cb[],...
    'Kg', @ob0, 1, 0, 0, 0, 0, 0@cb[],...
    ...
    'J',  @ob2, 1,-2, 0, 0, 0, 0@cb[],...
    'N',  @ob1, 1,-2, 0, 0, 0, 0@cb[]),
    ...
        \end{Verbatim}
      \end{onlyenv}
    \begin{onlyenv}<3>
      \begin{Verbatim}[commandchars=@\[\]]
  constants = struct(...
    'km',      {@ob1, 0, 0, 0, 0, 0, 0@cb[],1000},...
    'year',    {@ob0, 0, 1, 0, 0, 0, 0@cb[],31556926},...
    'AU',      {@ob1, 0, 0, 0, 0, 0, 0@cb[],149598000*1000},...
    'c',       {@ob1, 0,-1, 0, 0, 0, 0@cb[],299792458},...
    ...
        \end{Verbatim}
      \end{onlyenv}
    \end{column}
  \end{columns}
\end{frame}





% more ideas for functionality
% - sparsity
% - dependency analyzer
% - tracking memory (emulate Matlab's reference counting)





\begin{comment}
\begin{frame}[fragile,t]{Extending Functionality II - Units}
\begin{columns}[T]
  \begin{column}{5cm}
    \begin{onlyenv}<1>
      \begin{Verbatim}[commandchars=@\[\]]
aspect unit
  properties
    ...
  end

  patterns
    ...
  end

  actions
    ...
  end
end
        \end{Verbatim}
      \end{onlyenv}
    \begin{onlyenv}<2>
      \begin{Verbatim}[commandchars=@\[\]]
aspect unit
  @PYaG[properties]
  @PYaG[  ...]
  @PYaG[end]

  patterns
    ...
  end

  actions
    ...
  end
end
        \end{Verbatim}
      \end{onlyenv}
    \begin{onlyenv}<3>
      \begin{Verbatim}[commandchars=@\[\]]
properties
  units = struct(...
    'm',  [1, 0, 0, 0, 0, 0, 0],...
    'Kg', [0, 1, 0, 0, 0, 0, 0],...
    's',  [0, 0, 1, 0, 0, 0, 0],...
    ...
    'J',  [2, 1,-2, 0, 0, 0, 0],...
    'N',  [1, 1,-2, 0, 0, 0, 0]);
  end
  ...
end
        \end{Verbatim}
      \end{onlyenv}
    \begin{onlyenv}<4>
      \begin{Verbatim}[commandchars=@\[\]]
properties
  ...
  constants = struct(...
    'km',      {[1, 0, 0, 0, 0, 0, 0],1000},...
    'year',    {[0, 0, 1, 0, 0, 0, 0],31556926},...
    'lb',      {[0, 1, 0, 0, 0, 0, 0],0.45359237},...
    'inches',  {[1, 0, 0, 0, 0, 0, 0],0.0254},...
    'G',       {[3,-1,-2, 0, 0, 0, 0], 6.6730e-11},...
    'dozen',   {[0, 0, 0, 0, 0, 0, 0],12},...
    'AU',      {[1, 0, 0, 0, 0, 0, 0],149598000*1000},...
    'c',       {[1, 0,-1, 0, 0, 0, 0],299792458},...
    'KJ',      {[2, 1,-2, 0, 0, 0, 0],1000},...
  ...
end
        \end{Verbatim}
      \end{onlyenv}
    \begin{onlyenv}<5>
      \begin{Verbatim}[commandchars=@\[\]]
aspect unit
  properties
    ...
  end

@PYaG[  patterns]
@PYaG[    ...]
@PYaG[  end]

  actions
    ...
  end
end
        \end{Verbatim}
      \end{onlyenv}
    \begin{onlyenv}<6>
      \begin{Verbatim}[commandchars=@\[\]]
patterns
@PYaG[  allCalls : call(*());]

  plus : call(plus(*,*));
  mtimes : call(mtimes(*,*));
  ...

  loopheader : loophead(*);
  disp : call(disp);
end
        \end{Verbatim}
      \end{onlyenv}
    \begin{onlyenv}<7>
      \begin{Verbatim}[commandchars=@\[\]]
patterns
  allCalls : call(*());

@PYaG[  plus : call(plus(*,*));]
@PYaG[  mtimes : call(mtimes(*,*));]
@PYaG[  ...]

  loopheader : loophead(*);
  disp : call(disp);
end
        \end{Verbatim}
      \end{onlyenv}
    \begin{onlyenv}<8>
      \begin{Verbatim}[commandchars=@\[\]]
patterns
  allCalls : call(*());

  plus : call(plus(*,*));
  mtimes : call(mtimes(*,*));
  ...

@PYaG[  loopheader : loophead(*);]
@PYaG[  disp : call(disp);]
end
        \end{Verbatim}
      \end{onlyenv}
    \end{column}
  \end{columns}
\end{frame}
\end{comment}


%\end{frame}
%\begin{frame}{Domain-specific Outlook5}
%\begin{itemize}
%  \item several of our aspects override variables
%    \begin{itemize}
%    \item attach extra information to them
%    \end{itemize}
%  \item idea: incorporate this theme into a new class of aspects
%  \item one could have an aspect that annotates existing variables
%  \item operations should be specified on the annoations
%  \item a sort of abstract interpretation of annotations at runtime
%\end{itemize}
%\end{frame}

